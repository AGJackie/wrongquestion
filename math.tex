\documentclass{ctexart}
\input math.code.tex
\setdate{2020-09-15}

\begin{document}

\begin{mathques}{2020-09-05}{极限}
\begin{ques}
  极限$\lim_{n \to \infty} \left( \frac{n+1}{n-2} \right) ^n=$\mathblank.
\end{ques}
\begin{solu}
  \begin{align*}
    \lim_{n \to \infty} \left( \frac{n+1}{n-2} \right) ^n
    &= \lim_{n \to \infty} \frac{\left( 1 + \frac{1}{n} \right) ^n}{\left( 1 -
    \frac{2}{n} \right) ^n}\\
    & \mathhint{\lim_{n \to \infty} \left( 1 + \frac{1}{n} \right) ^n = e}
    \\
    & = \frac{\lim_{n \to \infty} \left( 1 + \frac{1}{n} \right) ^n}{\lim_{n
    \to \infty} \left( 1 - \frac{2}{n} \right) ^n}\\
    & = \frac{e}{e^{-2}} = e^3
  \end{align*}
\end{solu}
\end{mathques}

\begin{mathques}{2020-09-05}{极限}
\begin{ques}
  求极限$\lim_{n \to \infty} (\sqrt{n + \sqrt{n}} - \sqrt{n - \sqrt{n}})$.
\end{ques}
\begin{solu}
  \begin{align*}
    \lim_{n \to \infty} (\sqrt{n + \sqrt{n}} - \sqrt{n - \sqrt{n}})
    &= \lim_{n \to \infty} \frac{(n + \sqrt{n}) - (n - \sqrt{n})}{(n +
    \sqrt{n}) + (n - \sqrt{n})}\\
    &= \lim_{n \to \infty} \frac{2}{\sqrt{1 + \frac{1}{\sqrt{n}}} + \sqrt{1 -
    \frac{1}{\sqrt{n}}}}\\
    &= \frac{2}{\matherror{\sqrt{1} + \sqrt{1}}} = 1
  \end{align*}
\end{solu}
\end{mathques}

\begin{mathques}{2020-09-05}{极限}
\begin{ques}
  设$x_{n + 1} = \sqrt{2 + x_n} (n = 1, 2, \dotsc), x_1 = \sqrt{2}$, 证明
  $\lim_{n \to \infty} x_n$ 存在, 并求$\lim_{n \to \infty} x_n$.
\end{ques}
\begin{solu}
  \mathidea{单调有界数列必有极限}
  \begin{step}{证明单调性(3种方法)}
    \begin{enumerate}
      \item 令$f(x) = \sqrt{2 + x}$ 证明$f(x)$的单调性
      \item 作差$x_{n} - x_{n - 1} = \sqrt{2 + x_{n - 1}} - x_{n - 1} = \frac{2
        + x_{n - 1} - x_{n - 1}^2}{\sqrt{2 + x_{n - 1}} + x_{n - 1}}$\par
        $2 + x_{n - 1} - x_{n - 1}^2 > 2 + x_{n - 1} - 2x_{n - 1} = 2 - x_{n
        - 1} > 0$\par
        \mathidea{方法2需要先进行Step 2证明上界为2才行}
      \item 证明$x_{n + 1} - x_n$与$x_{n} - x_{n - 1}$同号\par
        $x_{n + 1} - x_n = \sqrt{2 + x_n} - \sqrt{2 + x_{n - 1}} = \frac{x_n
        - x_{n - 1}}{\sqrt{2 + x_n} + \sqrt{2 + x_{n - 1}}}$\par
        又$x_2 - x_1 = \sqrt{2 + \sqrt{2}} - \sqrt{2} > 0$,故$x_n$单调
    \end{enumerate}
  \end{step}
  \begin{step}{证明有界性}
    \begin{mathideabox}
      \itshape 当界值难以看出来时,可以先假定有界,得出有极限然后求出根据递
      推公式求出极限得出界值再反过来想怎么证明(见Step 3)
    \end{mathideabox}
    $x_1 = \sqrt{2} < 2$, 设$x_k < 2$, 则$x_{k + 1} = \sqrt{2 + x_k} < 2$,
    故$x_n < 2$有上界
  \end{step}
  \begin{step}{求极限}
    $x_n > 0$, 设$\lim_{n \to \infty} x_n = A > 0$, 则$A = \sqrt{2 + A} \Rightarrow A = 2$
  \end{step}
\end{solu}
\end{mathques}

\begin{mathques}(2){2020-09-08}{极限}
\begin{ques}
  求极限$\lim_{x \to 0^+} \frac{x^x - (\sin x)^x}{x^2 \ln (1 + x)}$.
\end{ques}
\begin{solu}
\begin{align*}
  \text{原式} & = -\matherror{\lim_{x \to 0^+} x^x} \lim_{x \to 0^+}
  \frac{\left( \frac{\sin x}{x} \right) ^x - 1}{x^3}\\
  &\mathhint{e^x - 1 \sim x (x \to 0)}\\
  &= -\lim_{x \to 0^+} \frac{\ln \frac{\sin x}{x}}{x^2}\\
  &= -\lim_{x \to 0^+} \frac{\ln \left( 1 + \frac{\sin x - x}{x} \right) }{x^2}
  \\
  &\mathhint{\ln (1+x) \sim x (x \to 0)}\\
  &= -\lim_{x \to 0^+} \frac{\sin x - x}{x^3}\\
  &= -\lim_{x \to 0^+} \frac{x - \frac{1}{3!} x^3 + o(x^3) - x}{x^3} = \frac{1}
  {6}
\end{align*}
\end{solu}
\end{mathques}

\begin{mathques}(2){2020-09-08}{极限}
\begin{ques}
  求极限$\lim_{x \to 0} \left( \frac{a_1^x + a_2^x + \dotsb + a_n^x}{n} \right)
  ^{\frac{n}{x}}$,其中$a_i > 0, i = 1, 2, \dotsc, n$.
\end{ques}
\begin{solu}
  因为$\lim _{x \to 0}a_i^x = 1$,所以原极限是“$1^{\infty}$”型未定式
  \mathmethod 使用\emph{洛必达法则}求极限
  \begin{align*}
    \lim_{x \to 0} \left( \frac{a_1^x + a_2^x + \dotsb + a_n^x}{n} \right)
    ^{\frac{n}{x}} &= \exp \left\{ \lim_{x \to 0}\frac{n}{x} \ln \left(
    \frac{a_1^x + a_2^x + \dotsb + a_n^x}{n} \right)  \right\}\\
    &\mathhint{\text{洛必达}\frac{0}{0}}\\
    &= \exp \left\{ \lim_{x \to 0} n \cdot \frac{a_1^x \ln a_1 + a_2^x \ln a_2
    + \dotsb + a_n^x \ln a_n}{a_1^x + a_2^x + \dotsb + a_n^x} \right\} \\
    &=a_1a_2\dotsi a_n
  \end{align*}
  \mathmethod 凑成第二个重要极限(“$1^{\infty}$”型未定式极限都可以凑成第二个重
  要极限
  \begin{align*}
    \lim_{x \to 0} \left( \frac{a_1^x + a_2^x + \dotsb + a_n^x}{n} \right)
    ^{\frac{n}{x}} &=
    \lim_{x \to 0} \left( 1 + \frac{a_1^x + a_2^x + \dotsb + a_n^x - n}{n}
    \right) ^{\frac{n}{a_1^x + a_2^x + \dotsb + a_n^x - n}
    \cdot \frac{a_1^x + a_2^x + \dotsb + a_n^x - n}{x}}
  \end{align*}
  其中
  \begin{align*}
    \lim_{x \to 0} \frac{a_1^x + a_2^x + \dotsb + a_n^x - n}{x}
    &= \ln (a_1a_2\dotsi a_n)\\
    \mathhint{\lim_{n \to \infty} \left( 1 + \frac{1}{n} \right) ^n = e}\\
    \lim_{x \to 0} \left( 1 + \frac{a_1^x + a_2^x + \dotsb + a_n^x - n}{n}
    \right) ^{\frac{n}{a_1^x + a_2^x + \dotsb + a_n^x - n}} &=e
  \end{align*}
\end{solu}
\end{mathques}

\begin{mathques}(2){2020-09-08}{极限}
\begin{ques}
  设函数$f(x) = \lim_{n \to \infty} \frac{1 + x}{1 + x^{2n}}$,讨论函数的间断
  点,其结论为(\quad).
  \begin{multichoice}
    \task 不存在间断点
    \task \answer{存在间断点$x = 1$}
    \task 存在间断点$x = 0$
    \task 存在间断点$x = -1$
  \end{multichoice}
\end{ques}
\begin{solu}
\begin{mathideabox}
  函数$f(x)$以$x$为自变量,但是在对$n$求极限的时候,$x$被看做常数,因此应根据
  $x$的不同取值求出对应的极限
\end{mathideabox}
当$|x| < 1$时, $\lim_{n \to \infty} x^{2n} = 0$,所以$f(x) = 1 + x$

当$|x| > 1$时,$\lim_{n \to \infty} \frac{1 + x}{1 + x^{2n}} = 0$

又$f(1) = 1, f(-1) = 0$,则
\[
  f(x) = \lim_{n \to \infty} \frac{1 + x}{1 + x^{2n}} =
  \begin{dcases}
    0,     & x \le -1 \\
    1 + x, & -1 <x < 1 \\
    1,     & x = 1 \\
    0,     & x > 1 \\
  \end{dcases}
\]
故$x = 1$为间断点
\end{solu}
\end{mathques}

\begin{mathques}(2){2020-09-09}{极限}
\begin{ques}
  设$f(\sin^2x) = \frac{x}{\sin x}$,则$f(x) = $\mathblank.
\end{ques}
\begin{solu}
  \answer{
    $\frac{\arcsin \sqrt{x}}{\sqrt{x}} \matherror{(x > 0)}$
  }\par
  设$u = \sin^2 x$,则$\sin x = \pm \sqrt{u}$.

  当$\sin x = \sqrt{u}$时,$x = \arcsin \sqrt{u}$

  当$\sin x = - \sqrt{u}$时,$\matherror{\sin(-x) = \sqrt{u}, x = -\arcsin
  \sqrt{u}}$.因此
  \begin{align*}
    f(u) &= \frac{\arcsin \sqrt{u}}{\sqrt{u}} \\
    f(x) &= \frac{\arcsin \sqrt{x}}{\sqrt{x}} (x > 0)
  \end{align*}
\end{solu}
\end{mathques}

\begin{mathques}{2020-09-07}{极限}
\begin{ques}
  设$f\left(x + \frac{1}{x}\right) = \frac{x + x^3}{1 + x^4}$,则$f(x) =
  $\mathblank
\end{ques}
\begin{solu}
  \answer{$\frac{x}{x^2 - 2}$}

  \[
    f\left(\matherror{ x + \frac{1}{x}}\right) = \frac{x + x^3}{1 + x^4} =
    \frac{x^2 \left( \matherror{\textstyle\frac{1}{x} + x} \right) }{x^2
    \left( \frac{1}{x^2} + x^2 \right) } = \frac{\frac{1}{x} + x}{\left(
  \matherror{\textstyle\frac{1}{x} + x} \right) ^2 - 2}
  \]
  因此$f(x) = \frac{x}{x^2 - 2}$
\end{solu}
\end{mathques}

\begin{mathques}(2){2020-09-09}{极限}
\begin{ques}
  求极限$\lim_{x \to 0^+} \frac{2e^{\frac{1}{x}} + e^{-\frac{1}{x}}}
  {e^{\frac{2}{x}} - e^{- \frac{1}{x}}}$.
\end{ques}
\begin{solu}
  \begin{align*}
    \lim_{x \to 0^+} \frac{2e^{\frac{1}{x}} + e^{-\frac{1}{x}}} {e^{\frac{2}
    {x}} - e^{- \frac{1}{x}}}
    &= \lim_{t \to +\infty} \frac{2e^t + e^{-t}}{e^{2t} - e^{-t}}
    \mathhint[r]{\text{令}t = \frac{1}{x}}\\
    &= \lim_{t \to +\infty} \frac{2 + e^{-2t}}{e^t - e^{-2t}}
    \mathhint[r]{\text{同除}e^t}\\
    &= 0
  \end{align*}
\end{solu}
\end{mathques}

\begin{mathques}(3){2020-09-11}{极限}
\begin{ques}
  求极限$\lim_{x \to 0} \frac{\sin x + x^2 \sin \frac{1}{x}}{(2 + x^2) \ln(1 +
  x)}$.
\end{ques}
\begin{solu}
  \begin{mathideabox}
    所给极限为“$\frac{0}{0}$”型,先进行\emph{等价无穷小代换},再拆分
  \end{mathideabox}
  \begin{align*}
    \lim_{x \to 0} \frac{\sin x + x^2 \sin \frac{1}{x}}{(2 + x^2) \ln(1 + x)}
    &= \lim_{x \to 0} \frac{1}{2 + x^2} \cdot \frac{\sin x + x^2 \sin \frac{1}
    {x}}{x} \mathhint[r]{\text{等价无穷小代换}}\\
    \mathhint[l]{\text{将非零因子}\frac{1}{2 + x^2}\text{单独求极限}}
    &= \frac{1}{2} \lim_{x \to 0} \left( \frac{\sin x}{x} + x \sin \frac{1}{x}
    \right) \\
    &= \frac{1}{2}
  \end{align*}
\end{solu}
\end{mathques}

\begin{mathques}{2020-09-11}{一元微分}
\begin{ques}
  设函数$f(x)$在$x = 0$处连续,且$\lim_{x \to 0} \frac{f(x^2)}{x^2} = 1$,则
  (\quad)
  \begin{multichoice}
    \task $f(0) = 0$且$f'_-(0)$存在
    \task $f(0) = 1$且$f'_-(0)$存在
    \task \answer{$f(0) = 0$且$f'_+(0)$存在}
    \task $f(0) = 1$且$f'_+(0)$存在
  \end{multichoice}
\end{ques}
\begin{solu}
  因为$f(x)$在$x = 0$处连续,且$\lim_{x \to 0} \frac{f(x^2)}{x^2} = 1$

  所以$\lim_{x \to 0} f(x^2) = 0$,即$f(0) = 0$

  从而有
  \begin{align*}
    \lim_{x \to 0} \frac{f(x^2)}{x^2}
    &= \matherror{\lim_{x \to 0} \frac{f(x^2) - f(0)}{x^2 - 0}}\\
    \mathhint[l]{t = x^2} &= \matherror{\lim_{t \to \color{red} 0^+}
    \frac{f(t) - f(0)}{t - 0}}\\
    &= f'_+(0)
  \end{align*}
\end{solu}
\end{mathques}

\begin{mathques}{2020-09-11}{一元微分}
\begin{ques}
  设函数$f(x) = \abs{x^2 - 1}\varphi (x)$,其中$\varphi (x) $在$x = 1$处连续,
  则$\varphi (1) = 0$是$f(x)$在$x = 1$处可导的(\quad)
  \begin{multichoice}
    \task \answer{充分必要条件}
    \task 充分但非必要条件
    \task 必要但非充分条件
    \task 既非充分也非必要条件
  \end{multichoice}
\end{ques}
\begin{solu}
  \mathidea{$\varphi(1) = 0 \Longrightarrow f(x)$在$x = 1$处可导}\par
  由$\varphi(1) = 0$可得$\mathhint{f(1) = 0}$
  \begin{align*}
    f'_+(1) &= \lim_{x \to 1^+} \frac{f(x) - f(1)}{x - 1}
    = \lim_{x \to 1^+} \frac{\abs{x^3 - 1}\varphi(x)}{x - 1}
    = \lim_{x \to 1^+} (x^2 + x + 1)\varphi(x) = 0\\
    f'_-(1) &= \lim_{x \to 1^-} \frac{f(x) - f(1)}{x - 1}
    = \lim_{x \to 1^-} \frac{\abs{x^3 - 1}\varphi(x)}{x - 1}
    = \lim_{x \to 1^-} (x^2 + x + 1)\varphi(x) = 0
  \end{align*}
  即$f'_+(1) = f'_-(1) = 0$,则$f'(1) = 0$

  \mathidea{$\varphi(1) = 0 \Longleftarrow f(x)$在$x = 1$处可导}\par
  设$f(x)$在$x = 1$处可导,因为$f(1) = 0$,所以
  \begin{align*}
    f'_+(1) &= \lim_{x \to 1^+} \frac{f(x) - f(1)}{x - 1}
    = \lim_{x \to 1^+} \frac{\abs{x^3 - 1}\varphi(x)}{x - 1}
    = \lim_{x \to 1^+} (x^2 + x + 1)\varphi(x) = 3\varphi(1)\\
    f'_-(1) &= \lim_{x \to 1^-} \frac{f(x) - f(1)}{x - 1}
    = \lim_{x \to 1^-} \frac{\abs{x^3 - 1}\varphi(x)}{x - 1}
    = \lim_{x \to 1^-} (x^2 + x + 1)\varphi(x) = -3\varphi(1)
  \end{align*}
  由$f'_+(1) = f'_-(1) = 0$得,$3\varphi(1) = -3\varphi(1)$,即$\varphi(1) = 0$
\end{solu}
\end{mathques}

\begin{mathques}{2020-09-11}{一元微分}
\begin{ques}
  设函数$y = f(x)$由方程$y = x\ln y$所确定,求$\dv{y}{x}$.
\end{ques}
\begin{solu}
\mathmethod 方程两边同时对$x$求导,得$y' = \ln y + x\cdot \frac{1}{y} \cdot
y'$,解得$y' = \frac{y \ln y}{y - x}$.
\mathmethod 令$F(x, y) = y - x\ln y$,则$F'_x = - \ln y, F'_y = 1 - \frac{x}{y}
$,故$\dv{y}{x} = - \frac{F'_x}{F'_y} = \frac{y\ln y}{y - x}$.
\end{solu}
\end{mathques}

\begin{mathques}(2){2020-09-13}{一元微分}
\begin{ques}
  已知$f'(x) = Ae^x$($A$为正常数),求$f(x)$的反函数的二阶导数.
\end{ques}
\begin{solu}
  \begin{mathideabox}[反函数定理]
    如果从$\mathbb{R}^n$的一个开集$\mathrm{U}$到$\mathbb{R}^n$的连续可微函数
    $F$的全微分在点$p$可逆($F$在点p的雅可比行列式不为零),则$F$在点$p$附近具
    有反函数,且
    \[
      J_{F^{-1}}(F(p)) = [J_F(p)]^{-1}
    \]
    其中$[\cdot]^{-1}$表示逆矩阵,$J_G(q)$是函数$G$在点$q$的雅可比矩阵。

    也可由链式法则推出:设$G,H$是两个函数,分别在$H(p)$和$p$处有全导数,则
    \[
      J_{G\circ H}(p) = J_G(H(p))\cdot J_H(p)
    \]
    设$G$为$F$,$H$为$F^{-1}$,$G\circ H$则为恒等函数,其雅可比矩阵也是单位矩
    阵。

    当$y$为$x$的一元函数时,其雅可比矩阵为$[\dd y / \dd x]$,即
    \[
      \dv{x}{y} = \frac{1}{\dv{y}{x}}
    \]
  \end{mathideabox}
  设$y = f(x)$,则$\dv{x}{y} = \frac{1}{f'(x)}$
  \begin{align*}
    \dv[2]{y}{x} &= \dv{y}(\dv{x}{y})\\
    &= \matherror{\dv{x}(\frac{1}{f'(x)}) \cdot \dv{x}{y}}\\
    &= \matherror{-\frac{f''(x)}{\qty(f'(x))^2} \cdot \frac{1}{f'(x)}}\\
    &= -\frac{f''(x)}{\qty(f'(x))^3} = - \frac{Ae^x}{(Ae^x)^3} = -\frac{1}
    {A^2e^{2x}}
  \end{align*}
\end{solu}
\end{mathques}

\begin{mathques}(2){2020-09-13}{一元微分}
\begin{ques}
  设可导函数$y = f(x)$有反函数$g(x)$,且$f(a) = 3, f'(a) = 1, f''(a) = 2$,求
  $g''(3)$.
\end{ques}
\begin{solu}
  \mathmethod
  由链式求导法则得$\qty(g \circ f(x))' = g'\qty(f(x)) \cdot f'(x)$,即
  \[
    \matherror{f'(x) \cdot g'(f(x)) = 1}
  \]
  两边同时对$x$求导,有
  \[
    f''(x)g'(f(x)) + \qty(f'(x))^2 \cdot g''(f(x)) = 0
  \]
  两边同乘$f'(x)$
  \[
    f''(x) + \qty(f'(x))^3 \cdot g''(f(x)) = 0
  \]
  将$x = a$代入得
  \[
    f''(a) + \qty(f'(a))^3 \cdot g''(f(a)) = 0
  \]
  即
  \[
    2 + g''(3) = 0
  \]
  则$g''(3) = -2$
  \mathmethod 公式法
  \[
    g''(3) = - \frac{f''(a)}{\qty(f(a))^3} = -2
  \]
  \begin{mathideabox}[公式推导过程]
    已知
    $\left\{\begin{aligned}
      \eval{y}_{x = a} &= 3\\
      \eval{\dv{y}{x}}_{x = a} &= 1\\
      \eval{\dv[2]{y}{x}}_{x = a} &= 2
    \end{aligned}\right.$
    , 求$\eval{\dv[2]{x}{y}}_{y = 3}$

    \begin{align*}
      \dv[2]{x}{y} &= \dv{y}(\dv{x}{y}) = \dv{x}(\dv{x}{y}) \cdot \dv{x}{y} \\
      &= \dv{x}(\frac{1}{\dv{y}{x}}) \cdot \frac{1}{\dv{y}{x}} \\
      &= - \frac{\dv{x}(\dv{y}{x})}{\qty(\dv{y}{x})^2} \cdot \frac{1}{\dv{y}{x}}\\
      &= - \frac{\dv[2]{y}{x}}{\qty(\dv{y}{x})^3}
    \end{align*}
    即$\eval{\dv[2]{x}{y}}_{y = 3} = \eval{- \frac{\dv[2]{y}{x}}{\qty(\dv{y}
    {x})^3}}_{x = a}$
  \end{mathideabox}
\end{solu}
\end{mathques}

\begin{mathques}{2020-09-13}{一元微分}
\begin{ques}
  设$f(x) =
  \begin{dcases}
    x^2 \sin \frac{\pi}{x}, & x < 0, \\
    A, & x = 0, \\
    ax^2 + b, & x > 0,\\
  \end{dcases}
  $求常数$A, a, b$的值,使$f(x)$在$x = 0$处可导,并求$f'(0)$.
\end{ques}
\begin{solu}
  \begin{mathideabox}[可导必然连续(反之不一定)]
    设函数$y = f(x)$上一点$x_0$,函数在这一点可导,即$f'(x_0) = \lim_{\Delta x
    \to 0} \frac{\Delta y}{\Delta x}存在$,其中
    \[
      \Delta y = f(x_0 + \Delta x) - f(x_0)
    \]
    则
    \begin{align*}
      \lim_{\Delta x \to 0} \Delta y &= \lim_{\Delta x \to 0} \qty(\frac{\Delta
      y}{\Delta x}\cdot \Delta x)\\
      &= f'(x_0) \cdot 0 = 0
    \end{align*}
    即函数$f(x)$在$x_0$处连续
  \end{mathideabox}
  \begin{mathideabox}[导数存在必然有左右导数相等]
    由极限的性质可知:
    \[
      \lim_{\Delta x \to 0} \frac{f(x_0 + \Delta x) - f(x_0)}{\Delta x}
      =\lim_{\Delta x \to 0^-} \frac{f(x_0 + \Delta x) - f(x_0)}{\Delta x}
      =\lim_{\Delta x \to 0^+} \frac{f(x_0 + \Delta x) - f(x_0)}{\Delta x}
    \]
  \end{mathideabox}
  可导必然连续,即
  \[
    \lim_{x \to 0^-} x^2 \frac{\pi}{x} = \lim_{x \to 0^+} (ax^2 + b) = A
  \]
  则$A = b = 0$.
  又
  \begin{align*}
    f'_-(0) &= \lim_{x \to 0^-} \frac{x^2 \sin \frac{\pi}{x} - 0}{x - 0} = 0\\
    f'_+(0) &= \lim_{x \to 0^+} \frac{ax^2 - 0}{x - 0} = 0
  \end{align*}
  所以,$a$可以为任意常数,且$f'(0) = 0$
\end{solu}
\end{mathques}

\begin{mathques}{2020-09-11}{一元微分}
\begin{ques}
  设函数$y = f(x)$由$
  \begin{dcases}
    x = \ln(1 + t^2) + 1,\\
    y = 2 \arctan t - (t + 1)^2,\\
  \end{dcases}
  $确定,求$\dv{y}{x}, \dv[2]{y}{x}$
\end{ques}
\begin{solu}
  \begin{align*}
    \dv{y}{t} &= \frac{2}{\matherror{1 + t^2}} - 2(t + 1)\\
    \dv{x}{t} &= \frac{2t}{1 + t^2}
  \end{align*}
  由链式法则,$\dv{y}{t} = \dv{y}{x}\cdot\dv{x}{t}$,则
  \begin{align*}
    \dv{y}{x} &= \frac{\dv{y}{t}}{\dv{x}{t}}
    = \frac{\frac{2}{1 + t^2} - 2(t + 1)}{\frac{2t}{1 + t^2}}
    = -(t^2 + t + 1)\\
    \dv[2]{y}{x} &= \dv{x}(\dv{y}{x})\\
    &= \dv{x}(-(t^2 + t + 1))\\
    &= \dv{t}(-(t^2 + t + 1)) \cdot \dv{t}{x}\\
    &= -\frac{1 + 2t}{\dv{x}{t}}\\
    &= -\frac{(1 + 2t)(1 + t^2)}{2t}
  \end{align*}
\end{solu}
\end{mathques}

\begin{mathques}{2020-09-13}{一元微分}
\begin{ques}
  设$f(x)$满足$f(0) = 0$,且$f'(0)$存在,求$\lim_{x \to 0} \frac{f(1 -
  \sqrt{\cos x})}{\ln(1 - x\sin x)}$
\end{ques}
\begin{solu}
  \begin{align*}
    \lim_{x \to 0} \frac{f(1 - \sqrt{\cos x})}{\ln(1 - x\sin x)}
    &=\lim_{x \to 0} \frac{f(1 - \sqrt{\cos x}) - f(0)}{(1 - x\sin x) - 0}
    \cdot \lim_{x \to 0} \frac{1 - \sqrt{\cos x}}{\ln(1 - x\sin x)}\\
    &=f'(0) \lim_{x \to 0} \frac{1 - \sqrt{\cos x}}{\ln(1 - x\sin x)}\\
    \mathidea{等价无穷小替换}&=f'(0)\lim_{x \to 0} \frac{1 - \cos x}{-x\sin x}
    \cdot \frac{1}{1 + \sqrt{\cos x}}\\
    &= -\frac{1}{2}f'(0)\lim_{x \to 0} \frac{1 - \cos x}{x^2}\\
    &= -\frac{1}{4}f'(0).
  \end{align*}
\end{solu}
\end{mathques}

\begin{mathques}{2020-09-12}{一元微分}
\begin{ques}
  设函数$f(x)$在$x = 0$处连续,且$\lim_{x \to 0} \frac{f(x)}{x}$存在,则(\quad)
  \begin{multichoice}
    \task $f(0) \neq 0$,但$f'(0)$可能不存在
    \task $f(0) = 0$,但$f'(0)$可能不存在
    \task \answer{$f'(0)$存在,但$f'(0)$不一定等于零}
    \task $f'(0)$存在,且必定有$f'(0) = 0$
  \end{multichoice}
\end{ques}
\begin{solu}
  \begin{mathideabox}[连续的定义]
    函数$f(x)$在点$p$处连续当且仅当$\lim_{x \to p} f(x) = f(p)$ (隐含着极限必
    须存在)
  \end{mathideabox}
  \begin{mathideabox}[重要结论]
    若$f(x)$在$x = 0$处连续,且$\lim_{x \to 0} \frac{f(x)}{x} = A$,则
    \begin{align*}
      f(0) &= 0\\
      f'(0) &= A
    \end{align*}
  \end{mathideabox}
  $f(x)$在$x = 0$处连续,则$\lim_{x \to 0} f(x)$存在且$\lim_{x \to 0} f(x) =
  f(0)$

  由$\lim_{x \to 0} \frac{f(x)}{x}$存在且分母极限为$0$可知,分子极限也为$0$,
  即$f(0) = 0$

  又
  \[
    f'(0) = \lim_{x \to 0} \frac{f(x) - f(0)}{x - 0} = \lim_{x \to 0}
    \frac{f(x)}{x}
  \]
  即$f'(0)$存在
\end{solu}
\end{mathques}

\begin{mathques}{2020-09-12}{一元微分}
\begin{ques}
  设$f(x) = x^a \abs{x}$,$a$为正整数,则函数$f(x)$在点$x = 0$处(\quad)
  \begin{multichoice}
    \task 不存在极限
    \task 存在极限,但不连续
    \task 连续但不可导
    \task \answer{可导}
  \end{multichoice}
\end{ques}
\begin{solu}
  $\lim_{x \to 0} f(x) = 0 = f(0)$,故$f(x)$在$x = 0$处连续
  \begin{align*}
    f'_+(0) &= \lim_{x \to 0^+} \frac{f(x) - f(0)}{x - 0} = \lim_{x \to 0^+}
    \frac{x^a \abs{x}}{x} = \matherror{\lim_{x \to 0^+} \frac{x^{a + 1}}{x} =
    0}\\
    f'_-(0) &= \lim_{x \to 0^-} \frac{f(x) - f(0)}{x - 0} = \lim_{x \to 0^-}
    \frac{x^a \abs{x}}{x} = \lim_{x \to 0^-} \frac{-x^{a + 1}}{x} = 0
  \end{align*}
  即$f'_+(0) = f'_-(0) = 0$,$f(x)$在点$x = 0$处可导,且导数为$0$.
\end{solu}
\end{mathques}

\begin{mathques}{2020-09-12}{一元微分}
\begin{ques}
  若$y = f(x)$可导,则当$\Delta x \to 0$时,$\Delta y - \dd y$为$\Delta x$的
  (\quad)
  \begin{multichoice}
    \task \answer{高阶无穷小}
    \task 低阶无穷小
    \task 同阶但不等价无穷小
    \task 等阶无穷小
  \end{multichoice}
\end{ques}
\begin{solu}
  \begin{mathideabox}[微分定义]
    设函数$y = f(x)$在区间$\mathcal{I}$有定义。对于$\mathcal{I}$内的一点$x_0$,
    当$x_0$变动到附近的$x_0 + \Delta x$(也在此区间)时,如果函数的增量$\Delta
    y = f\qty(x_0 + \Delta x - f(x_0))$可以表示为
    \[
      \Delta y = A\Delta x + o(\Delta x)
    \]
    其中$A$是不依赖于$\Delta x$的常数,$o(\Delta x)$是比$\Delta x$高阶的无穷小,
    则称函数$f(x)$在点$x_0$是可微的.

    $A\Delta x$称作函数在点$x_0$对应与自变量增量$\Delta x$的微分,记做$\dd
    y$,即$\dd y = A\Delta x$

    把自变量$x$的增量$\Delta x$称作自变量的微分,记做$\dd x$,即$\dd x =
    \Delta x$
  \end{mathideabox}
  由微分的定义可知$\Delta y - \dd y = o(\Delta x)$为$\Delta x$的高阶无穷小
\end{solu}
\end{mathques}

\begin{mathques}{2020-09-12}{一元微分}
\begin{ques}
设函数
\[
  f(x) =
  \begin{dcases}
    x^3\sin \frac{1}{x}, & x \neq 0,\\
    0, & x= 0,\\
  \end{dcases}
\]
讨论$f(x)$在$x = 0$处的可导性以及$f'(x)$在$x = 0$处的连续性.
\end{ques}
\begin{solu}
  \begin{mathideabox}
    证明$f(x)$在$x_0$处可导,即证明
    \begin{align*}
      f'(x_0) &= \lim_{\Delta x \to 0} \frac{\Delta y}{\Delta x}
      = \lim_{\Delta x \to 0} \frac{f(x_0 + \Delta x) - f(x_0)}{\Delta x}
      \\
      \text{或}f'(x_0) &= \lim_{x \to x_0} \frac{f(x) - f(x_0)}{x - x_0}
    \end{align*}
    存在
    \par\vspace{1em}
    证明$f(x)$在$x_0$处连续,即证明
    \[
      \lim_{x \to x_0} f(x) = f(x_0)
    \]
  \end{mathideabox}

  \matherror{\text{当}x = 0\text{时}},有
  \[
    f'(0) = \lim_{x \to 0} \frac{x^3 \sin \frac{1}{x} - 0}{x - 0} = 0,
  \]
  函数$f(x)$在$x = 0$处可导.

  \matherror{\text{当}x \neq 0\text{时}},有
  \[
    f'(x) = 3x^2 \sin \frac{1}{x} - x\cos \frac{1}{x}
  \]
  因为
  \[
    \lim_{x \to 0} f'(x) = \lim_{x \to 0} \qty(3x^2 \sin \frac{1}{x} - x\cos
    \frac{1}{x}) = 0 = f'(0),
  \]
  故函数f'(x)在$x = 0$处连续
\end{solu}
\end{mathques}
\end{document}
